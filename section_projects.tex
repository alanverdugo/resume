% Awesome Source CV LaTeX Template
%
% This template has been downloaded from:
% https://github.com/darwiin/awesome-neue-latex-cv
%
% Author:
% Christophe Roger
%
% Template license:
% CC BY-SA 4.0 (https://creativecommons.org/licenses/by-sa/4.0/)

%Section: Project at the top
\sectionTitle{Projects}{\faWrench}
%\renewcommand{\labelitemi}{$\bullet$}

\begin{projects}

%    \project
%        {January 2017}
%        {Conferences' attendees live tracking and analysis}
%        {IBM}
%        {October 2017}
%        {
%            \begin{itemize}
%                \item In less than a month, our team built a proof of concept 
%                    of a solution to the registration of attendees in 
%                    conferences, while also providing mobile applications and 
%                    an RFID tracking system used to identify the attendees' 
%                    participation in the venue.
%                \item Used the captured data to gain insights and create a 
%                    dashboard using the Watson Analytics platform.
%                \item Created reports by cleaning and processing the raw data 
%                    using Pandas and Matplotlib.
%                \item A fully-working demo was presented to IBM's global 
%                    leadership during the \textit{Interconnect} conference in 
%                    Las Vegas.
%            \end{itemize}
%        }
%        {
%            Watson Analytics,
%            Python,
%            Pandas,
%            Matplotlib,
%            MySQL,
%            SQLite
%        }
%
%    \emptySeparator

    \project
        {June 2016}
        {Cognitive Concierge}
        {IBM}
        {October 2016}
        {
            \begin{itemize}
                \item Our team trained and configured speech recognition 
                    patterns along with questions and answers in order to 
                    program a set of humanoid robots who could understand and 
                    answer natural language questions about IBM, Watson, cloud 
                    technologies, robotics and nearby locations.
                \item A \href{https://www.youtube.com/watch?v=yC0EOEdedzQ}
                    {\underline{fully-working demo}} was presented publicly 
                    during the \textit{World of Watson} conference in Las Vegas.
            \end{itemize}
        }
        {
            SoftBank Robotics' Nao and Pepper humanoid robots,
            Watson,
            Natural-language processing,
            Speech recognition
        }

    \emptySeparator

    \project
        {January 2016}
        {Hadoop Raspberry Pi Cluster}
        {IBM}
        {March 2016}
        {
            \begin{itemize}
                \item Built a fully-working 12-node Hadoop cluster with 
                    Raspberry Pies including setting up the environment in each 
                    node, testing it and presenting a demo to high management 
                    in order to demonstrate the feasibility of using 
                    Raspberries as an affordable cloud offering to entry-level 
                    clients.
                \item Documented the entire process and published it on the 
                    \href{https://developer.ibm.com/recipes/tutorials/building-a-hadoop-cluster-with-raspberry-pi/}
                    {\underline{IBM developerWorks site}}.
            \end{itemize}
        }
        {
            HDFS,
            Raspberry Pi
        }

    \emptySeparator

    \project
        {2020}
        {Gilbert (Educational open source mobile application)}
        {Independent}
        {2021}
        {
            \begin{itemize}
                \item Built a \href{https://play.google.com/store/apps/details?id=net.kippel.gilbert}{\underline{mobile application}}
                    aimed at helping college students to learn about electricity
                    and magnetism.
                \item Documented and
                    \href{https://github.com/alanverdugo/gilbert}
                    {\underline{published}} the entire application as free and open
                    source software.
            \end{itemize}
        }
        {
            Python,
            Kivy
        }

%    \emptySeparator
%    \project
%        {August 2016}
%        {Travel searcher}
%        {Independent}
%        {April 2018}
%        {
%            \begin{itemize}
%                \item By analyzing Google's travel data, built a wrapper and 
%                    notification system to inform me of affordable flights for 
%                    destinations, prices and schedules I decided.
%                \item Published all the code and documentation in 
%                    \href{https://github.com/alanverdugo/QPX}{\underline{my 
%                    github repository}}.
%            \end{itemize}
%        }
%        {
%            Python,
%            Google QPX API
%        }

\end{projects}
